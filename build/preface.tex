\documentclass[10pt]{article}
\usepackage[utf8]{inputenc}
\usepackage[papersize={17in, 11in}]{geometry}
\usepackage[absolute]{textpos}
\TPGrid[0.5in, 0.25in]{23}{24}
\usepackage{palatino}
\parindent=0pt
\parskip=12pt
\usepackage{nopageno}
\begin{document}

\begin{textblock}{23}(0, 1)
\center \huge PREFACE
\end{textblock}

\begin{textblock}{23}(0, 2.5)
\begin{center}
* * *
\end{center}
\end{textblock}

\begin{textblock}{6.5}(8.25, 3.5) \begin{center} \textit{Trees mark time in the
twists they make over the course of branches' growth. Arms rise sinewy in their
turnings-to-sky to fall earthward again \\ in a tracery of parts and of the
slow-moving shapes of time. } \end{center} \end{textblock}

\begin{textblock}{23}(0, 5.5)
\begin{center}
* * *
\end{center}
\end{textblock}

\begin{textblock}{23}(0, 7.5)
\textbf{The winds \& percussion are tranposed.} The E$\flat$ clarinet sounds a
minor third higher than written and the bass clarinet sounds a major ninth
lower than written. The xylophone sounds an octave higher than written. (But
note that the violin, viola and cello are all written at sounding pitch even
for the low notes of their scordatura.)

\textbf{Prioritization of tempo.} The proportions between tempi should be as
exact as possible (even though the choice of tempi are to some extent a matter
of the preferences of the ensemble and the acoustics of the hall). In addition,
the tempi of the very fast parts of the piece should be played as closely as
possible to the tempi written in the score: it is preferable to play the dense
figures in very fast parts of the piece as something of a blur rather than
slowing the tempi to attack each of the notes carefully. Speed and forcefulness
of tempo must take priority throughout the piece.

\textbf{Oboe \& clarinet.} All trills are color trills. Color fingerings are
given as circled Arabic numerals with greater numbers indicating greater
deviation from normal timbre.

\textbf{Piano and harpsichord.} The pianist is asked to switch between piano
and harpsichord throughout the piece. The two instruments should be positioned
right next to each other so that the pianist can switch rapidly. Register
settings for the harpsichord are encouraged even though none are given in the
score. The harpsichord should be amplified considerably. Piano and harpsichord
clusters are all chromatic.

\textbf{Percussion.} Seven instruments are required: (1.) a single crotale
(pitched in D$\natural$); (2.) a piece of slate scraped by an even smaller
piece of slate or another stone; (3.) a snare drum (played with the fingertips
and never with sticks); (4.) a large sponge whisked across the suface of a bass
drum; (5.) a single suspended cymbal (pitched as low as possible and played
with a soft yarn mallet); (6.) a tam-tam (as large as possible); (7.) and a
xylophone. A few of the switches between instrument are extremely fast. In
cases where it is not possible to effect the switch as quickly as written, the
last few notes of the previous material should be sacrificed so that the first
few notes of the next material begin on time. \textbf{The five-line staff always
indicates the xylophone.}

\textbf{Strings.} The the lowest strings of violin, viola and cello are all
detuned: the lowest string of the violin should be taken down one semitone to
F$\sharp$4; the lowest string of the viola should be taken down two semitones
to B$\flat$2; and the lowest string of the cello should be taken down three
semitones to A$\natural$1. In addition, the violinist is asked to play on a
large piece of slate at one point in the piece; the slate (and small stone used
for scraping in a circle) are to resemble those of the percussionist as closely
as possible. Tremolo are all fast and uncounted.

\end{textblock}

\begin{textblock}{23}(0, 23)

\textbf{Krummzeit} was written between May and August 2014 for Ensemble Mosaik.
The piece was premiered on the 4\textsuperscript{th} of October 2014 on the
campus of Harvard University by Ensemble Mosaik under the direction of Jonathan
Hepfer.

\end{textblock}

\end{document}