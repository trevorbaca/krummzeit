\documentclass[10pt]{article}
\usepackage[utf8]{inputenc}
\usepackage[papersize={17in, 11in}]{geometry}
\usepackage[absolute]{textpos}
\TPGrid[0.5in, 0.25in]{23}{24}
\usepackage{palatino}
\parindent=0pt
\parskip=12pt
\usepackage{nopageno}
\begin{document}

\begin{textblock}{23}(0, 1)
\center \huge PREFACE
\end{textblock}

\begin{textblock}{23}(0, 3)

\textit{The trees mark time in the twists they make over the course of their
branches' growth. Arms rise sinewy in their turnings-to-the sky to fall
earthward again in a tracery of parts and the slow-moving shapes of time. }

\textbf{The score is tranposed.} The E$\flat$ clarinet sounds a minor third
higher than written and the bass clarinet sounds a major ninth lower than
written. The xylophone sounds an octave higher than written.

\textbf{Prioritization of tempo.} The proportions between tempi should be as
exact as possible (even though the choice of exact tempi are to some extent a
matter of the preferences of the ensemble and the acoustics of the hall). In
addition, the tempi of the very fast parts of the piece should be played as
closely as possible to tempi written in the score: it is preferable to play the
dense figures in very fast parts of the piece as something of a blur (rather
than slowing the tempi to attack each of the notes carefully).

\textbf{Oboe \& clarinet.} All trills are color trills. Alternate color
fingerings are given as circled Arabic numerals with greater numbers indicating
greater deviation from the normal timbre. Glissandi may be executed by
embourchre or fingering.

\textbf{Piano and harpsichord.} The pianist is asked to switch between piano
and harpsichord throughout the piece. The two instruments should be positioned
right next to each other so that the pianist can switch rapidly between the
two. The pianist is encouraged to select register settings for the harpsichord
even though no such ettings are given in the score. The harpsichord should
amplified considerably. All clusters are chromatic clusters (of black and white
keys together).

\textbf{Percussion.} Seven instruments are required: (1.) a single crotale
(pitched in D$\natural$); (2.) a piece of slate scraped by an even smaller
piece of slate or another stone; (3.) a snare drum (played always as a quiet
roll with the fingertips and never with sticks); (4.) a large sponge whisked
across the suface of the largest possible bass drum; (5.) a single suspended
cymbal (pitched as low as possible and played always as an attackless roll with
the softest possible yarn mallet); (6.) a tam-tam (as large as possible); (7.)
and a xylophone.



\textbf{Krummzeit} was written between May and August 2014 and premiered on the
4\textsuperscript{th} of October 2014 on the campus of Harvard University by
Ensemble Mosaik under the direction of Eno Poppe. The piece is dedicated to
Ensemble Mosaik.

\end{textblock}

\end{document}